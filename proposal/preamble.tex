\usepackage{fontspec}

\setmainfont[Numbers=OldStyle]{Linux Libertine O}
\setsansfont[Numbers=OldStyle]{Linux Biolinum O}
\setmonofont[Scale=0.83]{Inconsolata}

\usepackage{polyglossia}
\setdefaultlanguage{english}

\usepackage{lipsum}

\usepackage{graphicx}
\usepackage[dvipsnames]{xcolor}
\usepackage{wrapfig}
\usepackage{subcaption}
\usepackage{lettrine}

\usepackage{amssymb, amsmath}

\usepackage{pdfpages}

\usepackage{microtype}

%% Propriétés du document PDF
\usepackage[unicode,colorlinks=true]{hyperref}

\hypersetup{
  pdfauthor={Dimitri Lozeve},
  pdftitle={Topological Data Analysis of time-dependent networks},
  pdfsubject={MSc project proposal},
  pdfkeywords={tda,network,project,msc},
  pdfpagemode=UseOutlines,
  % pdfpagelayout=TwoColumnRight,
  linkcolor=MidnightBlue,
  filecolor=MidnightBlue,
  urlcolor=MidnightBlue,
  citecolor=Green
}

%% Pour la classe memoir /!\

%% Marges
\setlrmarginsandblock{2.5cm}{3cm}{*}
%\setulmarginsandblock{4cm}{4cm}{*}
\checkandfixthelayout%

%% Numérotation des divisions logiques
\setsecnumdepth{subsection}
\maxsecnumdepth{subsection}

%% Profondeur de la ToC
\settocdepth{subsection}
\maxtocdepth{subsection}

%% Style des titres des divisions logiques
\setsecheadstyle{\Large\scshape}
\setsubsecheadstyle{\large\scshape}

%% Abstract
\abstractintoc%
\renewcommand{\abstractnamefont}{\normalfont\large\scshape}
\renewcommand{\abstracttextfont}{\normalfont\normalsize}

%% épigraphes
\setlength{\epigraphwidth}{0.5\textwidth}
\epigraphtextposition{flushleftright}

%% Couleurs
%\definecolor{purpletouch}{RGB}{103,30,117}
\definecolor{bleux}{RGB}{0,62,92}

\author{Dimitri Lozeve}
\date{February 15, 2018}
\title{MSc project proposal\\
  \Large Topological Data Analysis of time-dependent networks}






%%% Local Variables:
%%% mode: latex
%%% TeX-master: "proposal"
%%% End:

