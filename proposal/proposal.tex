\documentclass[article,a4paper,11pt,openany,extrafontsizes]{memoir}

\usepackage{fontspec}

\setmainfont[Numbers=OldStyle]{Linux Libertine O}
\setsansfont[Numbers=OldStyle]{Linux Biolinum O}
\setmonofont[Scale=0.83]{Inconsolata}

\usepackage{polyglossia}
\setdefaultlanguage[variant=british]{english}

\usepackage{lipsum}

\usepackage{graphicx}
\usepackage[dvipsnames]{xcolor}
\usepackage{wrapfig}
\usepackage{subcaption}
\usepackage{lettrine}

\usepackage{amssymb, amsmath}
\usepackage{amsthm}

\theoremstyle{plain}
\newtheorem{thm}{Theorem}[chapter]
\newtheorem{lem}[thm]{Lemma}
\newtheorem{cor}[thm]{Corollary}
\newtheorem{prop}[thm]{Proposition}
\theoremstyle{definition}
\newtheorem{defn}{Definition}[chapter]
\newtheorem{expl}{Example}[chapter]
\theoremstyle{remark}
\newtheorem*{rem}{Remark}
\newtheorem*{note}{Note}
\newtheorem*{notation}{Notation}

\usepackage{pdfpages}

\usepackage{microtype}

%% Propriétés du document PDF
\usepackage[unicode,colorlinks=true]{hyperref}

\hypersetup{
  pdfauthor={Dimitri Lozeve},
  pdftitle={Topological Data Analysis of Temporal Networks},
  pdfsubject={A dissertation submitted un partial fulfilment of the degree of Master of Science in Applied Statistics},
  pdfkeywords={dissertation,stats,msc,tda,networks},
  pdfpagemode=UseOutlines,
  pdfpagelayout=TwoColumnRight,
  linkcolor=MidnightBlue,
  filecolor=MidnightBlue,
  urlcolor=MidnightBlue,
  citecolor=Green
}

%% Pour la classe memoir /!\

%% Marges
\setlrmarginsandblock{3cm}{3cm}{*}
\setulmarginsandblock{3cm}{3cm}{*}
\checkandfixthelayout%

%% Numérotation des divisions logiques
\setsecnumdepth{subsection}
\maxsecnumdepth{subsection}

%% Profondeur de la ToC
\settocdepth{subsection}
\maxtocdepth{subsection}

%% Style des titres des divisions logiques
\setsecheadstyle{\Large\scshape}
\setsubsecheadstyle{\large\scshape}

%% Abstract
\abstractintoc%
\renewcommand{\abstractnamefont}{\normalfont\large\scshape}
\renewcommand{\abstracttextfont}{\normalfont\normalsize}

%% épigraphes
\setlength{\epigraphwidth}{0.5\textwidth}
\epigraphtextposition{flushleftright}

%% Couleurs
%\definecolor{purpletouch}{RGB}{103,30,117}
\definecolor{bleux}{RGB}{0,62,92}

\author{Dimitri Lozeve}
\date{September 2018}
\title{Topological Data Analysis of Temporal Networks}







%%% Local Variables:
%%% mode: latex
%%% TeX-master: "dissertation"
%%% End:


\usepackage[backend=biber,style=ieee,url=false,arxiv=abs]{biblatex}
\addbibresource{proposal.bib}

\tightlists%

\begin{document}

\maketitle

% \subsection*{Title}

% Topological Data Analysis of Time-dependent Networks

\subsection*{Supervisors}

Dr Heather Harrington (Mathematical Institute) and Dr Gesine Reinert
(Department of Statistics)

\subsection*{Description}

Topological Data Analysis (TDA)~\cite{chazal_introduction_2017,
  oudot_persistence_2015, carlsson_topology_2009,
  edelsbrunner_computational_2010} is a family of techniques gaining
an increasing importance in the analysis and visualization of
high-dimensional data in machine learning applications.

In this project, we will apply TDA techniques and persistent homology
to time-dependent networks, in order to understand how the topological
structure evolves over time in complex multilayer
networks~\cite{kivela_multilayer_2014, porter_dynamical_2014}.

There are two ways of obtaining time-dependent networks. Network data
is available easily in many contexts: social networks and biological
processes are two examples of systems evolving over time and that can
be modelled as a graph. For instance, in social networks, links in ego
networks have already been studied in the context of
time-dependency~\cite{tabourier_predicting_2016}.

The other large category is time series. It is possible to use a
similarity measure to build a network from a set of time series taken
from the same physical process. Although it could be applied to any
set of time series, this has already been studied in the case of
coupled oscillators (such as Kuramoto
oscillators)~\cite{stolz_persistent_2017, schaub_graph_2016}. It is
thus easy to find relevant datasets or to generate interesting data
from physical simulations.

It is then possible to apply existing TDA and persistent homology
techniques to the networks, taking into account the temporal
dimension. Certain methods have already been implemented in
topological data analysis libraries~\cite{tierny_topology_2017,
  maria_gudhi_2014}, although they would have to be adapted to network
data, and applied repeatedly to each time step. There is also a wide
range of methods to explore, from the choice of the similarity
measure, to the choice of filtration (in order to build a simplicial
complex on the network), to the representation of topological
structure. Each of these choices has a great influence on the final
interpretation of the data, and may need to be adapted to each system.

\subsection*{Prerequisite courses/knowledge}

\begin{itemize}
\item SM7 Probability and Statistics for Network Analysis
\item Topological Data Analysis and Persistent
  Homology\footnote{\url{http://www.enseignement.polytechnique.fr/informatique/INF556/}}
\end{itemize}

\subsection*{Computing required?}

Yes

\subsection*{Data available?}

Yes

%%\nocite{*}
% \bibliographystyle{ieeetr}
% \bibliography{proposal}
\printbibliography%

\end{document}

%%% Local Variables:
%%% mode: latex
%%% TeX-master: t
%%% End:
