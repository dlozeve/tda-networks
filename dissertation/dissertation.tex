\documentclass[a4paper,11pt,openany,extrafontsizes]{memoir}

\usepackage{fontspec}

\setmainfont{Linux Libertine O}
\setsansfont{Linux Biolinum O}
\setmonofont[Scale=0.83]{Inconsolata}

\usepackage{polyglossia}
\setdefaultlanguage[variant=british]{english}

\usepackage{graphicx}
\usepackage[dvipsnames]{xcolor}
\usepackage{wrapfig}
\usepackage{subcaption}
\usepackage{lettrine}

\usepackage{amssymb, amsmath}
\usepackage{amsthm}

\theoremstyle{plain}
\newtheorem{thm}{Theorem}[chapter]
\newcommand{\thmautorefname}{theorem}
\newtheorem{lem}[thm]{Lemma}
\newcommand{\lemautorefname}{lemma}
\newtheorem{cor}[thm]{Corollary}
\newcommand{\corautorefname}{corollary}
\newtheorem{prop}[thm]{Proposition}
\newcommand{\propautorefname}{proposition}
\theoremstyle{definition}
\newtheorem{defn}{Definition}[chapter]
\newcommand{\defnautorefname}{definition}
\newtheorem{expl}{Example}[chapter]
\newcommand{\explautorefname}{example}
\theoremstyle{remark}
\newtheorem*{rem}{Remark}
\newtheorem*{note}{Note}
\newtheorem*{notation}{Notation}

\DeclareMathOperator{\dgm}{dgm}

\usepackage{tikz-network}
\usepackage{tikz}
\usetikzlibrary{patterns,backgrounds,positioning,chains,lindenmayersystems}

\usepackage[style=numeric-comp,backref,url=false]{biblatex}
\bibliography{TDA,temporalgraphs}

\usepackage{pdfpages}

\usepackage{microtype}

%% Propriétés du document PDF
\usepackage[unicode,colorlinks=true]{hyperref}

\hypersetup{
  pdfauthor={Dimitri Lozeve},
  pdftitle={Topological Data Analysis of Temporal Networks},
  pdfsubject={A dissertation submitted un partial fulfilment of the degree of Master of Science in Applied Statistics},
  pdfkeywords={dissertation,stats,msc,tda,networks},
  pdfpagemode=UseOutlines,
  pdfpagelayout=TwoColumnRight,
  linkcolor=MidnightBlue,
  filecolor=MidnightBlue,
  urlcolor=MidnightBlue,
  citecolor=Green
}

%% Pour la classe memoir /!\

%% Marges
\setlrmarginsandblock{3cm}{3cm}{*}
\setulmarginsandblock{3cm}{3cm}{*}
\checkandfixthelayout%

%% Numérotation des divisions logiques
\setsecnumdepth{subsection}
\maxsecnumdepth{subsection}

%% Profondeur de la ToC
\settocdepth{subsection}
\maxtocdepth{subsection}

%% Style des titres des divisions logiques
\setsecheadstyle{\Large\scshape}
\setsubsecheadstyle{\large\scshape}

%% Abstract
\abstractintoc%
\renewcommand{\abstractnamefont}{\normalfont\large\scshape}
\renewcommand{\abstracttextfont}{\normalfont\normalsize}

%% épigraphes
\setlength{\epigraphwidth}{0.5\textwidth}
\epigraphtextposition{flushleftright}

%% Couleurs
%\definecolor{purpletouch}{RGB}{103,30,117}
\definecolor{bleux}{RGB}{0,62,92}

\author{Dimitri Lozeve}
\date{September 2018}
\title{Topological Data Analysis of Temporal Networks}







%%% Local Variables:
%%% mode: latex
%%% TeX-master: "dissertation"
%%% End:


\usepackage[firstpage]{draftwatermark}


\begin{document}

\pagestyle{plain}
\tightlists%

\begin{titlingpage}
  \begin{center}
    \vspace{1cm}
    \textsf{\Huge{University of Oxford}}\\
    \vspace{1cm}
    \includegraphics{branding/beltcrest.png}\\
    \vspace{2cm}
    \Huge{\thetitle}\\
    \vspace{2cm}
    \large{by\\[14pt]\theauthor\\[8pt]St Catherine's College}
    \vfill
    %% Inkscape L-system
    %% [C]++[C]++[C]++[C]++[C]
    %% B=DA++EA----CA[-DA----BA]++;C=+DA--EA[---BA--CA]+;D=-BA++CA[+++DA++EA]-;E=--DA++++BA[+EA++++CA]--CA;A=
    \begin{tikzpicture}
      \pgfdeclarelindenmayersystem{Penrose}{
        \symbol{M}{\pgflsystemdrawforward}
        \symbol{N}{\pgflsystemdrawforward}
        \symbol{O}{\pgflsystemdrawforward}
        \symbol{P}{\pgflsystemdrawforward}
        \symbol{A}{\pgflsystemdrawforward}
        \symbol{+}{\pgflsystemturnright}
        \symbol{-}{\pgflsystemturnleft}
        \rule{M->OA++PA----NA[-OA----MA]++}
        \rule{N->+OA--PA[---MA--NA]+}
        \rule{O->-MA++NA[+++OA++PA]-}
        \rule{P->--OA++++MA[+PA++++NA]--NA}
        \rule{A->}
      }
      \draw[lindenmayer system={Penrose, axiom=[N]++[N]++[N]++[N]++[N],
        order=2, angle=36, step=4pt}]
      lindenmayer system;
    \end{tikzpicture}
    % \vspace{2.2cm}
    \vfill
    \large{A dissertation submitted in partial fulfilment of the degree of\\
      Master of Science in Applied Statistics}\\
    \vspace{.5cm}
    \large{\emph{Department of Statistics,\\ 24--29 St Giles, Oxford, OX1 3LB}}\\
    \vspace{1cm}
    \large{\thedate}
  \end{center}
\end{titlingpage}

%\chapterstyle{hangnum}
%\chapterstyle{ell}
%\chapterstyle{southall}
\chapterstyle{wilsondob}

\frontmatter

\cleardoublepage%

\chapter*{Declaration of authorship}

\emph{This my own work (except where otherwise indicated).}\\[2cm]

\begin{center}
  Date \hspace{.5\linewidth} Signature
\end{center}


\cleardoublepage%

\begin{abstract}
  Abstract here
\end{abstract}

\cleardoublepage%

\chapter*{Acknowledgements}%
\label{cha:acknowledgements}

Thank you!

\cleardoublepage%

\tableofcontents
\listoffigures
% \listoftables

\clearpage

\mainmatter%

\chapter{Introduction}%
\label{cha:introduction}


\chapter{Graphs and Temporal Networks}%
\label{cha:temporal-networks}

\section{Definition and basic properties}%
\label{sec:defin-basic-prop}

In this section, we introduce the notion of temporal networks (or
temporal graphs). This is a complex notion, with many concurrent
definitions and interpretations.

After clarifying the notations, we restate the standard definition of
a non-temporal graph.

\begin{notation}
  \begin{itemize}
  \item $\mathbb{N}$ is the set of non-negative natural numbers
    $0,1,2,\ldots$ 
  \item $\mathbb{N}^*$ is the set of positive integers $1,2,\ldots$
  \item $\mathbb{R}$ is the set of real numbers.
    $\mathbb{R}_+ = \{x\in\mathbb{R} \;|\; x\geq 0\}$, and
    $\mathbb{R}_+^* = \{x\in\mathbb{R} \;|\; x>0\}$.
\end{itemize}
\end{notation}

\begin{defn}[Graph]
  A \emph{graph} is a couple $G = (V, E)$, where $V$ is a set of
  \emph{nodes} (or \emph{vertices}), and $E \subseteq V\times V$ is a
  set of \emph{edges}. A \emph{weighted graph} is defined by
  $G = (V, E, w)$, where $w : E\mapsto \mathbb{R}_+^*$ is called the
  \emph{weight function}.
\end{defn}

We also define some basic concepts that we will need later to build
simplicial complexes on graphs.

\begin{defn}[Clique]
  A \emph{clique} is a set of nodes where each pair is adjacent. That
  is, a clique $C$ of a graph $G = (V,E)$ is a subset of $V$ such that
  for all $i,j\in C, i \neq j \implies (i,j)\in E$. A clique is said
  to be \emph{maximal} if it cannot be augmented by any node, such
  that the resulting set of nodes is itself a clique.
\end{defn}

Temporal networks can be defined in the more general framework of
\emph{multilayer networks}~\cite{kivela_multilayer_2014}. However,
this definition is much too general for our simple applications, and
we restrict ourselves to edge-centric time-varying
graphs~\cite{casteigts_time-varying_2012}. In this model, the set of
nodes is fixed, but edges can appear or disappear at different times.

In this study, we restrict ourselves to discrete time stamps. Each
interaction is taken to be instantaneous.
%% TODO note about data collection, oversampling,
%% duration of interactions

\begin{defn}[Temporal network]
  A \emph{temporal network} is a tuple
  $G = (V, E, \mathcal{T}, \rho)$, where:
  \begin{itemize}
  \item $V$ is a set of nodes,
  \item $E\subseteq V\times V$ is a set of edges,
  \item $\mathbb{T}$ is the \emph{temporal domain} (often taken as
    $\mathbb{N}$ or any other countable set), and
    $\mathcal{T}\subseteq\mathbb{T}$ is the \emph{lifetime} of the
    network,
  \item $\rho: E\times\mathcal{T}\mapsto\{0,1\}$ is the \emph{presence
      function}, which determines whether an edge is present in the
    network at each time stamp.
  \end{itemize}
  The \emph{available times} of an edge are the set
  $\mathcal{I}(e) = \{t\in\mathcal{T}: \rho(e,t)=1\}$.
\end{defn}

Temporal networks can also have weighted edges. In this case, it is
possible to have constant weights (edges can only appear or disappear
over time, and always have the same weight), or time-varying
weights. In the latter case, we can set the domain of the presence
function to be $\mathbb{R}_+$ instead of $\{0,1\}$, where by
convention a 0 weight corresponds to an absent edge.

\begin{defn}[Additive and dismantling temporal
  networks]\label{defn:additive}
  A temporal network is said to be \emph{additive} if for all $e\in E$
  and $t\in\mathcal{T}$, if $\rho(e,t)=1$, then for all
  $t'>t, \rho(e, t') = 1$. An additive network can only gain edges
  over time.

  A temporal network is said to be \emph{dismantling} if for all
  $e\in E$ and $t\in\mathcal{T}$, if $\rho(e,t)=0$, then for all
  $t'>t, \rho(e, t') = 0$. An dismantling network can only lose edges
  over time.
\end{defn}

\section{Examples of applications}%
\label{sec:exampl-appl}

%% TODO

\section{Network partitioning}%
\label{sec:network-partitioning}

%% TODO clarify, organise, references

Temporal networks are a very active research subject, leading to
multiple interesting problems. The additional time dimension adds a
significant layer of complexity that cannot be adequately treated by
the common methods on static graphs.

Moreover, data collection can lead to large amount of noise in
datasets. Combined with large dataset sized due to the huge number of
data points for each node in the network, temporal graphs cannot be
studied effectively in their raw form. Recent advances have been made
to fit network models to rich but noisy
data~\cite{newman_network_2018}, generally using some variation on the
expectation-maximization (EM) algorithm.

One solution that has been proposed to study such temporal data has
been to \emph{partition} the time scale of the network into a sequence
of smaller, static graphs, representing all the interactions during a
short interval of time. The approach consists in subdividing the
lifetime of the network in \emph{sliding windows} of a given length.
We can then ``flatten'' the temporal network on each time interval,
keeping all the edges that appear at least once (or adding their
weights in the case of weighted networks).

This partitioning is sensitive to two parameters: the length of each
time interval, and their overlap. Of those, the former is the most
important: it will define the \emph{resolution} of the study. If it is
too small, too much noise will be taken into account; if it is too
large, we will lose important information. There is a need to find a
compromise, which will depend on the application and on the task
performed on the network. In the case of a classification task to
determine periodicity, it will be useful to adapt the resolution to
the expected period: if we expect week-long periodicity, a resolution
of one day seems reasonable.

Once the network is partitioned, we can apply any statistical learning
task on the sequence of static graphs. In this study, we will focus on
classification of time steps. This can be used to detect periodicity,
outliers, or even maximise temporal communities.

%% TODO Talk about partitioning methods?

\chapter{Topological Data Analysis and Persistent Homology}%
\label{cha:tda-ph}

%% TODO references

\section{Basic constructions}%
\label{sec:basic-constructions}

\subsection{Homology}%
\label{sec:homology}

Our goal is to understand the topological structure of a metric
space. For this, we can use \emph{homology}, which consists of
associating a vector space $H_i(X)$ to a metric space $X$ and a
dimension $i$. The dimension of $H_i(X)$ gives us the number of
$i$-dimensional components in $X$: the dimension of $H_0(X)$ is the
number of path-connected components in $X$, the dimension of $H_1(X)$
is the number of holes in $X$, and the dimension of $H_2(X)$ is the
number of voids.

Crucially, these vector spaces are robust to continuous deformation of
the underlying metric space (they are \emph{homotopy
  invariant}). However, computing the homology of an arbitrary metric
space can be extremely difficult. It is necessary to approximate it in
a structure that would be both combinatorial and topological in
nature.

\subsection{Simplicial complexes}%
\label{sec:simplicial-complexes}

To understand the topological structure of a metric space, we need a
way to decompose it in smaller pieces that, when assembled, conserve
the overall organisation of the space. For this, we use a structure
called a \emph{simplicial complex}, which is a kind of
higher-dimensional generalization of a graph.

The building blocks of this representation is the \emph{simplex},
which is the convex hull of an arbitrary set of points. Examples of
simplices include single points, segments, triangles, and tetrahedrons
(in dimensions 0, 1,, 2, and 3 respectively).

\begin{defn}[Simplex]
  A \emph{$k$-dimensional simplex} $\sigma = [x_0,\ldots,x_k]$ is the
  convex hull of the set $\{x_0,\ldots,x_k\} \in \mathbb{R}^d$, where
  $x_0,\ldots,x_k$ are affinely independent. $x_0,\ldots,x_k$ are
  called the \emph{vertices} of $\sigma$, and the simplices defined by
  the subsets of $\{x_0,\ldots,x_k\}$ are called the \emph{faces} of
  $\sigma$.
\end{defn}

\begin{figure}[ht]
  \centering
  \begin{subfigure}[b]{.3\linewidth}
    \centering
    \begin{tikzpicture}
      \tikzstyle{point}=[circle,thick,draw=black,fill=blue!30,%
      inner sep=0pt,minimum size=15pt]
      \node (a)[point] at (0,0) {a};
    \end{tikzpicture}
    \caption{Single vertex}
  \end{subfigure}%
  % 
  \begin{subfigure}[b]{.3\linewidth}
    \centering
    \begin{tikzpicture}
      \tikzstyle{point}=[circle,thick,draw=black,fill=blue!30,%
      inner sep=0pt,minimum size=15pt]
      \node (a)[point] at (0,0) {a};
      \node (b)[point] at (1.4,2) {b};
      
      \begin{scope}[on background layer]
        \draw[fill=blue!15] (a.center) -- (b.center) -- cycle;
      \end{scope}
    \end{tikzpicture}
    \caption{Segment}
  \end{subfigure}%
  % 
  \begin{subfigure}[b]{.3\linewidth}
    \centering
    \begin{tikzpicture}
      \tikzstyle{point}=[circle,thick,draw=black,fill=blue!30,%
      inner sep=0pt,minimum size=15pt]
      \node (a)[point] at (0,0) {a};
      \node (b)[point] at (1.4,2) {b};
      \node (c)[point] at (2.8,0) {c};
      
      \begin{scope}[on background layer]
        \draw[fill=blue!15] (a.center) -- (b.center) -- (c.center) -- cycle;
      \end{scope}
    \end{tikzpicture}
    \caption{Triangle}
  \end{subfigure}%
  % 
  \caption{Examples of simplices}%
  \label{fig:simplex}
\end{figure}


We then need a way to meaningfully combine these basic building blocks
so that the resulting object can adequately reflect the topological
structure of the metric space.

\begin{defn}[Simplicial complex]
  A \emph{simplicial complex} is a collection $K$ of simplices such
  that:
  \begin{itemize}
  \item any face of a simplex of $K$ is a simplex of $K$
  \item the intersection of two simplices of $K$ is either the empty
    set, or a common face, or both.
  \end{itemize}
\end{defn}

\begin{figure}[ht]
  \centering
  \begin{tikzpicture}
    \tikzstyle{point}=[circle,thick,draw=black,fill=blue!30,%
      inner sep=0pt,minimum size=10pt]
      \node (a)[point] {};
      \node (b)[point,above right=1.4cm and 1cm of a] {};
      \node (c)[point,right=2cm of a] {};
      \node (d)[point,above right=.4cm and 2cm of b] {};
      \node (e)[point,above right=.4cm and 2cm of c] {};
      \node (f)[point,below right=.7cm and 1.3cm of c] {};
      \node (g)[point,right=2cm of d] {};
      \node (h)[point,below right=.4cm and 1.5cm of e] {};
      
      \begin{scope}[on background layer]
        \draw[fill=blue!15] (a.center) -- (b.center) -- (c.center) -- cycle;
        \draw (b) -- (d) -- (g);
        \draw (c.center) -- (e.center) -- (f.center) -- cycle;
        \draw (d) -- (e) -- (h);
      \end{scope}

      \node (1)[point,right=2cm of g] {};
      \node (2)[point,above right=.5cm and 1cm of 1] {};
      \node (3)[point,below right=.5cm and 1cm of 2] {};
      \node (4)[point,below left=1cm and .3cm of 3] {};
      \node (5)[point,below right=1cm and .3cm of 1] {};
      \node (6)[point,below left=1cm and .1cm of 5] {};
      \node (7)[point,below right=1cm and .1cm of 4] {};
      \node (8)[point,below right=.7cm and .7cm of 6] {};

      \begin{scope}[on background layer]
        \draw[fill=green!15] (1.center) -- (2.center) -- (3.center) -- (4.center) -- (5.center) -- cycle;
        \draw (1) -- (4) -- (2) -- (5) -- (3) -- (1);
        \draw[fill=blue!15] (6.center) -- (7.center) -- (8.center) -- cycle;
        \draw (5) -- (6) -- (4) -- (7);
      \end{scope}
  \end{tikzpicture}
  \caption{Example of a simplicial complex that has two connected
    components, two 3-simplices, and one 5-simplex.}%
  \label{fig:simplical-complex}
\end{figure}

The notion of simplicial complex is closely related to that of a
hypergraph. One important distinction lies in the fact that a subset
of a hyperedge is not necessarily a hyperedge itself.

Using these definitions, we can define homology on simplicial
complexes. %% TODO add reference for more details/do it myself?

\subsection{Filtrations}%
\label{sec:filtrations}

%% TODO rewrite it using the Cech complex as an introductory example,
%% to understand the problem with scale

If we consider that a simplicial complex is a kind of
``discretization'' of a subset of a metric space, we realise that
there must be an issue of \emph{scale}. For our analysis to be
invariant under small perturbations in the data, we need a way to find
the optimal scale parameter to capture the adequate topological
structure, without taking into account some small perturbations, nor
ignoring some important smaller features.

One possible solution to these problems is to consider all scales at
once. This is the objective of \emph{filtered simplicial complexes}.

\begin{defn}[Filtration]\label{defn:filt}
  A \emph{filtered simplicial complex}, or simply a \emph{filtration},
  $K$ is a sequence ${(K_i)}_{i\in I}$ of simplicial complexes such
  that:
  \begin{itemize}
  \item for any $i, j \in I$, if $i < j$ then $K_i \subseteq K_j$,
  \item $\bigcup_{i\in I} K_i = K$.
  \end{itemize}
\end{defn}

\section{Persistent Homology}%
\label{sec:persistent-homology}

We can now compute the homology for each step in a filtration. This
leads to the notion of \emph{persistent
  homology}~\cite{carlsson_topology_2009,zomorodian_computing_2005},
which gives all the information necessary to establish the topological
structure of a metric space at multiple scales.

\begin{defn}[Persistent homology]
  The \emph{$p$-th persistent homology} of a simplicial complex
  $K = {(K_i)}_{i\in I}$ is the pair
  $(\{H_p(K_i)\}_{i\in I}, \{f_{i,j}\}_{i,j\in I, i\leq j})$, where
  for all $i\leq j$, $f_{i,j} : H_p(K_i) \mapsto H_p(K_j)$ is induced
  by the inclusion map $K_i \mapsto K_j$.
\end{defn}

The functions $f_{i,j}$ allow one to link generators in each
successive homology space in a filtration. Because each generator
corresponds to a topological feature (connected component, hole, void,
and so on, depending on the dimension $p$), we can determine whether
it survives in the next step of the filtration. We can also determine
when each feature is born and when it dies (if it dies at all). The
couples of intervals (birth time, death time) depends on the choice of
basis for each homology space $H_p(K_i)$. However, by the Fundamental
Theorem of Persistent Homology~\cite{zomorodian_computing_2005}, we
can choose basis vectors in each homology space such that the
collection of half-open intervals is well-defined and unique. This
construction is called a \emph{barcode}~\cite{carlsson_topology_2009}.

\section{Topological summaries: barcodes and persistence diagrams}%
\label{sec:topol-summ}

%% TODO need more context

To interpret the results of the persistent-homology computation, we
need to compare the output for a particular data set to a suitable
null model. For this, we need some kind of similarity measure between
barcodes and a way to evaluate the statistical significance of the
results.

One possible approach is to define a space in which we can project
barcodes and study their geometric properties. One such space is the
space of \emph{persistence
  diagrams}~\cite{edelsbrunner_computational_2010}.

\begin{defn}[Multiset]
  A \emph{multiset} $M$ is the couple $(A, m)$, where $A$ is the
  \emph{underlying set} of $M$, formed by its distinct elements, and
  $m : A\mapsto\mathbb{N}^*$ is the \emph{multiplicity function}
  giving the number of occurrences of each element of $A$ in $M$.
\end{defn}

\begin{defn}[Persistence diagrams]
  A \emph{persistence diagram} is the union of a finite multiset of
  points in $\overline{\mathbb{R}}^2$ with the diagonal
  $\Delta = \{(x,x) \;|\; x\in\mathbb{R}^2\}$, where every point of
  $\Delta$ has infinite multiplicity.
\end{defn}

One adds the diagonal $\Delta$ for technical reasons. It is convenient
to compare persistence diagrams by using bijections between them, so
persistence diagrams must have the same cardinality.

In some cases, the diagonal in the persistence diagrams can also
facilitate comparisons between diagrams, as points near the diagonal
correspond to short-lived topological features, so they are likely to
be caused by small perturbations in the data.

One can build a persistence diagram from a barcode by taking the union
of the multiset of (birth, death) couples with the diagonal
$\Delta$. \autoref{fig:pipeline} summarises the entire pipeline.

\begin{figure}[ht]
  \centering
  \begin{tikzpicture}
    \tikzstyle{pipelinestep}=[rectangle,thick,draw=black,inner sep=5pt,minimum size=15pt]
    \node (data)[pipelinestep] {Data};
    \node (filt)[pipelinestep,right=1cm of data] {Filtered complex};
    %% \node (barcode)[pipelinestep,right=1cm of filt] {Barcodes};
    \node (dgm)[pipelinestep,right=1cm of filt] {Persistence diagram};
    \node (interp)[pipelinestep,right=1cm of dgm] {Interpretation};

    \draw[->] (data.east) -- (filt.west);
    %% \draw[->] (filt.east) -- (barcode.west);
    \draw[->] (filt.east) -- (dgm.west);
    \draw[->] (dgm.east) -- (interp.west);
  \end{tikzpicture}

  \caption{Persistent homology pipeline}%
  \label{fig:pipeline}
\end{figure}

One can define an operator $\dgm$ as the first two steps in the
pipeline. It constructs a persistence diagram from a subset of a
metric space, via persistent homology on a filtered complex.

We can now define several distances on the space of persistence
diagrams.

\begin{defn}[Wasserstein distance]
  The \emph{$p$-th Wasserstein distance} between two diagrams $X$ and
  $Y$ is
  \[ W_p[d](X, Y) = \inf_{\phi:X\mapsto Y} \left[\sum_{x\in X} {d\left(x, \phi(x)\right)}^p\right] \]
  for $p\in [1,\infty)$, and:
  \[ W_\infty[d](X, Y) = \inf_{\phi:X\mapsto Y} \sup_{x\in X} d\left(x,
      \phi(x)\right) \] for $p = \infty$, where $d$ is a distance on
  $\mathbb{R}^2$ and $\phi$ ranges over all bijections from $X$ to
  $Y$.
\end{defn}

\begin{defn}[Bottleneck distance]
  The \emph{bottleneck distance} is defined as the infinite
  Wasserstein distance where $d$ is the uniform norm:
  $d_B = W_\infty[L_\infty]$.
\end{defn}

The bottleneck distance is symmetric, non-negative, and satisfies the
triangle inequality. However, it is not a true distance, as one can
come up with two distinct diagrams with bottleneck distance 0, even
on multisets that do not touch the diagonal $\Delta$.

\section{Stability}%
\label{sec:stability}

One of the most important aspects of topological data analysis is that
it is \emph{stable} with respect to small perturbations in the
data. More precisely, the second step of the pipeline
in~\autoref{fig:pipeline} is Lipschitz with respect to a suitable
metric on filtered complexes and the bottleneck distance on
persistence
diagrams~\cite{cohen-steiner_stability_2007,chazal_persistence_2014}. First,
we define a distance between subsets of a metric
space~\cite{oudot_persistence_2015}.

\begin{defn}[Hausdorff distance]
  Let $X$ and $Y$ be subsets of a metric space $(E, d)$. The
  \emph{Hausdorff distance} is defined by
  \[ d_H(X,Y) = \max \left[ \sup_{x\in X} \inf_{y\in Y} d(x,y),
      \sup_{y\in Y} \inf_{x\in X} d(x,y) \right]. \]
\end{defn}

We can now give an appropriate stability
property~\cite{cohen-steiner_stability_2007,chazal_persistence_2014}.

\begin{prop}
  Let $X$ and $Y$ be subsets in a metric space. We have
  \[ d_B(\dgm(X),\dgm(Y)) \leq d_H(X,Y). \]
\end{prop}

\section{Algorithms and implementations}%
\label{sec:algor-impl}

%% TODO
\cite{morozov_dionysus:_2018,bauer_ripser:_2018,reininghaus_dipha_2018,maria_gudhi_2014}

\section{Discussion}%
\label{sec:discussion}

%% TODO

%% information thrown away in filtrations and in PH


\chapter{Topological Data Analysis on Networks}%
\label{cha:topol-data-analys}

\section{Persistent homology for networks}%
\label{sec:pers-homol-netw}

We now consider the problem of applying persistent homology to network
data. An undirected network is already a simplicial complex of
dimension 1. However, this is not sufficient to capture enough
topological information; we need to introduce higher-dimensional
simplices. One method is to project the nodes of a network onto a
metric space~\cite{otter_roadmap_2017}, thereby transforming the
network data into a point-cloud data. For this, we need to compute the
distance between each pair of nodes in the network (e.g.\ with the
shortest-path distance). This also requires the network to be
connected. %% TODO defn of connected?

Another common method, for weighted networks, is called the
\emph{weight rank-clique filtration}
(WRCF)~\cite{petri_topological_2013}, which filters a network based
on weights. The procedure works as follows:
\begin{enumerate}
\item Consider the set of all nodes, without any edge, to be
  filtration step~0.
\item Rank all edge weights in decreasing order $\{w_1,\ldots,w_n\}$.
\item At filtration step $t$, keep only the edges whose weights are
  larger than or equal to $w_t$, thereby creating an unweighted graph.
\item Define the maximal cliques of the resulting graph to be
  simplices.
\end{enumerate}

At each step of the filtration, we construct a simplicial complex
based on cliques; this is called a \emph{clique
  complex}~\cite{zomorodian_tidy_2010}. The result of the algorithm is
itself a filtered simplicial complex (\autoref{defn:filt}), because a
subset of a clique is necessarily a clique itself, and the same is
true for the intersection of two cliques.

This leads to a first possibility for applying persistent homology to
temporal networks. It is possible to segment the lifetime of a network
into sliding windows, creating a time-independent graph on each window
by retaining only the edges available during the time interval. We can
then apply WRCF on each graph in the sequence, obtaining a filtered
complex for each window, to which we can then apply persistent
homology.

This method can quickly become very computationally expensive, as
finding all maximal cliques (e.g.\ using the Bron--Kerbosch algorithm)
is a complicated problem, with an optimal computational complexity of
$\mathcal{O}\big(3^{n/3}\big)$~\cite{tomita_worst-case_2006}. In
practice, one often restrict the search to cliques of dimension less
than or equl to a certain bound $d_M$. With this restriction, the new
simplicial complex is homologically equivalent to the original: they
have the same homology groups up to $H_{d_M-1}$.

%% TODO rewrite this paragraph
This method is sensitive to the choice of sliding windows on the time
scale. The width and the overlap of the windows can completely change
the networks created and their topological features. Too small a
window, and the network becomes too small to have any significant
topological properties, too large, and we lose important information
in the evolution of the network over time.

\section{Zigzag persistence}%
\label{sec:zigzag-persistence}

The standard algorithm to compute persistent homology
(see~\autoref{sec:persistent-homology}) relies on the fact that
filtrations (see~\autoref{defn:filt}) are nested sequences of
simplicial complexes:
\[ \cdots \subseteq K_{i-1} \subseteq K_i \subseteq K_{i+1} \subseteq
  \cdots \]

One can now create an independent filtration (e.g.\ with WRCF) for
each time step. The issue is that the topological features will be
orthogonal to the time dimension.

Another possibility is to create a filtration along the time
dimension. The issue in this case is that the sequence is no longer
nested (except for additive or dismantling temporal networks,
see~\autoref{defn:additive}).

The solution to consider the time dimension is provided by
\emph{zigzag persistence}~\cite{carlsson_zigzag_2009}, which allows
one to compute persistence on alternating nested sequences:
\[ \cdots \supseteq K_{i-1} \subseteq K_i \supseteq K_{i+1} \subseteq
  \cdots \]

This sequence can in turn be computed from a temporal network by
computing the union of each pair of consecutive time steps,
constructing an alternating sequence.

Zigzag persistence is a special case of the more general concept of
\emph{multi-parameter
  persistence}~\cite{carlsson_theory_2009,dey_computing_2014}, where
filtrations can encompass multiple parameters.

%% Note about libraries implementing zigzag persistence: Dionysus

\chapter{Persistent Homology for Machine-Learning Applications}%
\label{cha:pers-homol-mach}

The output of persistent homology is not directly usable by most
statistical methods. For example, barcodes and persistence diagrams,
which are multisets of points in $\overline{\mathbb{R}}^2$, are not
elements of a metric space in which one can perform statistical
computations.

The distances between persistence diagrams defined
in~\autoref{sec:topol-summ} allow one to compare different
outputs. From a statistical perspective, it is possible to use a
generative model of simplicial complexes and to use a distance between
persistence diagrams to measure the similarity of our observations
with this null model~\cite{adler_persistent_2010}. This would
effectively define a metric space of persistence diagrams. It is even
possible to define some statistical summaries (means, medians,
confidence intervals) on these
spaces~\cite{turner_frechet_2014,munch_probabilistic_2015}.

%% TODO REFERENCES

The issue with this approach is that metric spaces do not offer enough
algebraic structure to be amenable to most machine-learning
techniques. One of the most recent development in the study of
topological summaries has been to find mappings between the space of
persistence diagrams and Banach spaces.

\section{Vectorization methods}%
\label{sec:vect-meth}

%% TODO

\subsection{Persistence landscapes}

Persistence landscapes~\cite{bubenik_statistical_2015} give a way to
project barcodes to a space where it is possible to add them
meaningfully. It is then possible to define means of persistence
diagrams, as well as other summary statistics.

The function mapping a persistence diagram to a persistence landscape
is \emph{injective}, but no explicit inverse exists to go back from a
persistence landscape to the corresponding persistence
diagram. Moreover, a mean of persistence landscapes does not
necessarily have a corresponding persistence diagram.

\begin{defn}[Persistence landscape]
  The persistence landscape of a diagram $D = \{(b_i,d_i)\}_{i=1}^n$
  is the set of functions $\lambda_k: \mathbb{R} \mapsto \mathbb{R}$,
  for $k\in\mathbb{N}$, such that
  \[ \lambda_k(x) = k\text{-th largest value of } \{f_{(b_i,
      d_i)}(x)\}_{i=1}^n, \] (and $\lambda_k(x) = 0$ if the $k$-th
  largest value does not exist), where $f_{(b,d)}$ is a
  piecewise-linear function defined by:
  \[ f_{(b,d)} =
    \begin{cases}
      0,& \text{if }x \notin (b,d),\\
      x-b,& \text{if }x\in (b,\frac{b+d}{2}),\\
      -x+d,& \text{if }x\in (\frac{b+d}{2},d)\,.
    \end{cases}
  \]
\end{defn}

Moreover, one can show that persistence landscapes are stable with
respect to the $L^p$ distance, and that the Wasserstein and bottleneck
distances are bounded by the $L^p$
distance~\cite{bubenik_statistical_2015}. We can thus view the
landscapes as elements of a Banach space in which we can perform the
statistical computations.

\subsection{Persistence images}

\cite{adams_persistence_2017}

\subsection{Tropical and arctic semirings}

\cite{kalisnik_tropical_2018}

\section{Kernel-based methods}%
\label{sec:kernel-based-methods}

\subsection{Persistent scale-space kernel}

\cite{reininghaus_stable_2015,kwitt_statistical_2015}

\subsection{Persistence weighted-Gaussian kernel}

\cite{kusano_kernel_2017}

\subsection{Sliced Wasserstein kernel}

\cite{carriere_sliced_2017}

\section{Comparison}%
\label{sec:comparison}

\chapter{Conclusions}%
\label{cha:conclusions}




\backmatter%

% \nocite{*}
\printbibliography%

\end{document}



%%% Local Variables:
%%% mode: latex
%%% TeX-master: t
%%% End:
