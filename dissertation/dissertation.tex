\documentclass[a4paper,11pt,openany,extrafontsizes]{memoir}

\usepackage{fontspec}

\setmainfont{Linux Libertine O}
\setsansfont{Linux Biolinum O}
\setmonofont[Scale=0.83]{Inconsolata}

\usepackage{polyglossia}
\setdefaultlanguage[variant=british]{english}

\usepackage{graphicx}
\usepackage[dvipsnames]{xcolor}
\usepackage{wrapfig}
\usepackage{subcaption}
\usepackage{lettrine}

\usepackage{amssymb, amsmath}
\usepackage{amsthm}

\theoremstyle{plain}
\newtheorem{thm}{Theorem}[chapter]
\newcommand{\thmautorefname}{theorem}
\newtheorem{lem}[thm]{Lemma}
\newcommand{\lemautorefname}{lemma}
\newtheorem{cor}[thm]{Corollary}
\newcommand{\corautorefname}{corollary}
\newtheorem{prop}[thm]{Proposition}
\newcommand{\propautorefname}{proposition}
\theoremstyle{definition}
\newtheorem{defn}{Definition}[chapter]
\newcommand{\defnautorefname}{definition}
\newtheorem{expl}{Example}[chapter]
\newcommand{\explautorefname}{example}
\theoremstyle{remark}
\newtheorem*{rem}{Remark}
\newtheorem*{note}{Note}
\newtheorem*{notation}{Notation}

\DeclareMathOperator{\dgm}{dgm}

\usepackage{tikz-network}
\usepackage{tikz}
\usetikzlibrary{patterns,backgrounds,positioning,chains,lindenmayersystems}

\usepackage[style=numeric-comp,backref,url=false]{biblatex}
\bibliography{TDA,temporalgraphs}

\usepackage{pdfpages}

\usepackage{microtype}

%% Propriétés du document PDF
\usepackage[unicode,colorlinks=true]{hyperref}

\hypersetup{
  pdfauthor={Dimitri Lozeve},
  pdftitle={Topological Data Analysis of Temporal Networks},
  pdfsubject={A dissertation submitted un partial fulfilment of the degree of Master of Science in Applied Statistics},
  pdfkeywords={dissertation,stats,msc,tda,networks},
  pdfpagemode=UseOutlines,
  pdfpagelayout=TwoColumnRight,
  linkcolor=MidnightBlue,
  filecolor=MidnightBlue,
  urlcolor=MidnightBlue,
  citecolor=Green
}

%% Pour la classe memoir /!\

%% Marges
\setlrmarginsandblock{3cm}{3cm}{*}
\setulmarginsandblock{3cm}{3cm}{*}
\checkandfixthelayout%

%% Numérotation des divisions logiques
\setsecnumdepth{subsection}
\maxsecnumdepth{subsection}

%% Profondeur de la ToC
\settocdepth{subsection}
\maxtocdepth{subsection}

%% Style des titres des divisions logiques
\setsecheadstyle{\Large\scshape}
\setsubsecheadstyle{\large\scshape}

%% Abstract
\abstractintoc%
\renewcommand{\abstractnamefont}{\normalfont\large\scshape}
\renewcommand{\abstracttextfont}{\normalfont\normalsize}

%% épigraphes
\setlength{\epigraphwidth}{0.5\textwidth}
\epigraphtextposition{flushleftright}

%% Couleurs
%\definecolor{purpletouch}{RGB}{103,30,117}
\definecolor{bleux}{RGB}{0,62,92}

\author{Dimitri Lozeve}
\date{September 2018}
\title{Topological Data Analysis of Temporal Networks}







%%% Local Variables:
%%% mode: latex
%%% TeX-master: "dissertation"
%%% End:


\usepackage[firstpage]{draftwatermark}


\begin{document}

\pagestyle{plain}
\tightlists%

\begin{titlingpage}
  \begin{center}
    \vspace{1cm}
    \textsf{\Huge{University of Oxford}}\\
    \vspace{1cm}
    \includegraphics[scale=.8]{Stats_Logo.png}\\
    \vspace{2cm}
    \Huge{\thetitle}\\
    \vspace{2cm}
    \large{by\\[14pt]\theauthor\\[8pt]St Catherine's College}\\
    % \vspace{2.2cm}
    \vfill
    \large{A dissertation submitted in partial fulfilment of the degree of Master of Science in Applied Statistics}\\
    \vspace{.5cm}
    \large{\emph{Department of Statistics, 24--29 St Giles,\\Oxford, OX1 3LB}}\\
    \vspace{1cm}
    \large{\thedate}
  \end{center}
\end{titlingpage}

%\chapterstyle{hangnum}
%\chapterstyle{ell}
%\chapterstyle{southall}
\chapterstyle{wilsondob}

\frontmatter

\cleardoublepage%

\chapter*{Declaration of authorship}

\emph{This my own work (except where otherwise indicated).}\\[2cm]

\begin{center}
  Date \hspace{.5\linewidth} Signature
\end{center}


\cleardoublepage%

\begin{abstract}
  Abstract here
\end{abstract}

\cleardoublepage%

\chapter*{Acknowledgements}%
\label{cha:acknowledgements}

Thank you!

\cleardoublepage%

\tableofcontents*
\listoffigures*
\listoftables*

\clearpage

\mainmatter%

\chapter{Introduction}%
\label{cha:introduction}



\chapter{Topological Data Analysis and Persistent Homology}%
\label{cha:tda-ph}

\section{Homology}%
\label{sec:homology}

Our goal is to understand the topological structure of a metric
space. For this, we can use \emph{homology}, which consists in
associating for a metric space $X$ and a dimension $i$ a vector space
$H_i(X)$. The dimension of $H_i(X)$ will give us the number of
$i$-dimensional components in $X$: the dimension of $H_0(X)$ is the
number of path-connected components in $X$, the dimension of $H_1(X)$
is the number of holes in $X$, and the dimension of $H_2(X)$ is the
number of voids.

Crucially, these vector spaces are robust to continuous deformation of
the underlying metric space (they are \emph{homotopy
  invariant}). However, computing the homology of an arbitrary metric
space can be extremely difficult. It is necessary to approximate it in
a structure that would be both combinatorial and topological in
nature.

\section{Simplicial Complexes}%
\label{sec:simplicial-complexes}

In order to understand the topological structure of a metric space, we
need a way to decompose it in smaller pieces which, when assembled,
conserve the overall organisation of the space. For this, we use a
structure called a \emph{simplicial complex}, which is a kind of
higher-dimensional generalization of graphs.

The building blocks of this representation will be \emph{simplices},
which are simply the convex hull of an arbitrary set of
points. Examples of simplices include single points, segments,
triangles, and tetrahedrons (in dimensions 0, 1,, 2, and 3
respectively).

\begin{defn}[Simplex]
  The \emph{$k$-dimensional simplex} $\sigma = [x_0,\ldots,x_k]$ is
  the convex hull of the set $\{x_0,\ldots,x_k\} \in \mathbb{R}^d$,
  where $x_0,\ldots,x_k$ are affinely independent. $x_0,\ldots,x_k$
  are called the \emph{vertices} of $\sigma$, and the simplices
  defined by the subsets of $\{x_0,\ldots,x_k\}$ are called the
  \emph{faces} of $\sigma$.
\end{defn}

We then need a way to combine these basic building blocks meaningfully
so that the resulting object can adequately reflect the topological
structure of the metric space.

\begin{defn}[Simplicial complex]
  A \emph{simplicial complex} is a collection $K$ of simplices such
  that:
  \begin{itemize}
  \item any face of a simplex of $K$ is a simplex of $K$
  \item the intersection of two simplices of $K$ is either the empty
    set or a common face or both.
  \end{itemize}
\end{defn}

%% TODO figure with examples of simplicial complexes

The notion of simplicial complex is closely related to that of a
hypergraph. The important distinction lies in the fact that a subset
of a hyperedge is not necessarily a hyperedge itself.

Using these definitions, we can define homology on simplicial
complexes. %% TODO add reference for more details/do it myself?

\section{Filtrations}%
\label{sec:filtrations}

If we consider that a simplicial complex is a kind of
``discretization'' of a metric space, we realise that there must be an
issue of \emph{scale}. For our analysis to be invariant under small
perturbations in the data, we need a way to find the optimal scale
parameter to capture the adequate topological structure, without
taking into account some small perturbations, nor ignoring some
important smaller features.

%% TODO rewrite using the Cech filtration as an example?

The ideal solution to these problems is to consider all scales at
once: this is the objective of \emph{filtered simplical complexes}.

\begin{defn}[Filtration]
  A \emph{filtered simplicial complex}, or simply a \emph{filtration},
  $K$ is a sequence ${(K_i)}_{i\in I}$ of simplicial complexes such
  that:
  \begin{itemize}
  \item for any $i, j \in I$, if $i < j$ then $K_i \subseteq K_j$,
  \item $\bigcup_{i\in I} K_i = K$.
  \end{itemize}
\end{defn}

\section{Persistent Homology}%
\label{sec:persistent-homology}

We can now compute the homology for each step in a filtration. This
leads to the notion of \emph{persistent homology}, which gives us all
the information necessary to establish the topological structure of
the metric space at multiple scales.

\begin{defn}[Persistent homology]
  The \emph{$p$-th persistent homology} of a simplicial complex
  $K = {(K_i)}_{i\in I}$ is the pair
  $(\{H_p(K_i)\}_{i\in I}, \{f_{i,j}\}_{i,j\in I, i\leq j})$, where
  for all $i\leq j$, $f_{i,j} : H_p(K_i) \mapsto H_p(K_j)$ is induced
  by the inclusion map $K_i \mapsto K_j$.
\end{defn}

The functions $f_{i,j}$ allow us to link generators in each successive
homology space in the filtration. Since each generator correspond to a
topological feature (connected component, hole, void, etc, depending
on the dimension $p$), we can determine whether it survives in the
next step of the filtration. We can now determine when each feature is
born and when it dies (if it dies at all). This representation will be
dependent on the choice of basis for each homology space
$H_p(K_i)$. However, by the Fundamental Theorem of Persistent
Homology, we can choose base vectors in each homology space such that
the collection of half-open intervals is well-defined and unique. This
construction is called a \emph{barcode}.
%% TODO references for the Fundamental Theorem

\section{Topological summaries: barcodes and persistence diagrams}%
\label{sec:topol-summ}

In order to interpret the results of the persistent homology
computation, we need to compare the output for a particular data set
to a suitable null model. For this, we need some kind of a similarity
measure between barcodes and a way to evaluate the statistical
significance of the results.

One possible approach for this is to define a space in which we can
project barcodes and study their geometric
properties. \emph{Persistence diagrams} are an example of such a
space.

\begin{defn}[Persistence diagrams]
  A \emph{persistence diagram} is the union of a finite multiset of
  points in $\bar{\mathbb{R}}^2$ zith the diagonal
  $\Delta = \{(x,x) \;|\; x\in\mathbb{R}^2\}$, where every point of
  $\Delta$ has infinite multiplicity.
\end{defn}

The diagonal $\Delta$ is added to facilitate comparisons between
diagrams, as points near the diagonal correspond to short-lived
topological feature, thus likely to be caused by small perturbations
in the data.

We can now define several distances on the space of persistence
diagrams.

\begin{defn}[Wasserstein distance]
  The \emph{$p$-th Wasserstein distance} between two diagrams $X$ and
  $Y$ is
  \[ W_p[d](X, Y) = \inf_{\phi:X\mapsto Y} \left[\sum_{x\in X} {d\left(x, \phi(x)\right)}^p\right] \]
  for $p\in [1,\infty)$, and
  \[ W_\infty[d](X, Y) = \inf_{\phi:X\mapsto Y} \sup_{x\in X} d\left(x,
      \phi(x)\right) \] for $p = \infty$, where $d$ is a distance on
  $\mathbb{R}^2$ and $\phi$ ranges over all bijections from $X$ to
  $Y$.
\end{defn}

\begin{defn}[Bottleneck distance]
  The \emph{bottleneck distance} is defined as the infinite
  Wasserstein distance with $d$ the uniform norm:
  $d_B = W_\infty[L_\infty]$.
\end{defn}

Since the bottleneck distance is by far the most commonly used, we
will focus on it in the following. It is symmetric, non-negative, and
satisfies the triangle inequality. However, it is not a true distance,
as it is fairly straightforward to come up with two distinct diagrams
at bottleneck distance zero, even on multisets not touching the
diagonal $\Delta$.

\section{Stability}%
\label{sec:stability}



\chapter{Temporal Networks}%
\label{cha:temporal-networks}

\section{Definition and basic properties}%
\label{sec:defin-basic-prop}

In this section, we will introduce the notion of temporal networks or
graphs. This is a complex notion, with many concurrent definitions and
interpretations. First, we restate the standard definition of a
non-temporal, static graph.

\begin{defn}[Graph]
  A \emph{graph} is a couple $G = (V, E)$, where $V$ is a finite set
  of \emph{nodes} (or \emph{vertices}), and $E \subseteq V\times V$ is
  a set of \emph{edges}. A \emph{weighted graph} is defined by
  $G = (V, E, w)$, where $w : E\mapsto \mathbb{R}_+$ is fcalled the
  \emph{weight function}.
\end{defn}

We also define some basic concepts that will be needed later on to
build simplicial complexes on graphs.

\begin{defn}[Clique]
  A \emph{clique} is a set of nodes where each pair is connected. That
  is, a clique $C$ of a graph $G = (V,E)$ is a subset of $V$ such that
  $\forall i,j\in C, i \neq j \implies (i,j)\in E$. A clique is said
  to be \emph{maximal} if it cannot be augmented by any node.
\end{defn}

Temporal networks are defined in the more general framework of
\emph{multilayer networks}. However, this definition is much too
general for our simple applications, and we restrict ourselves to
edge-centric time-varying graphs. In this model, the set of nodes is
fixed and doesn't change over time, whereas edges can appear or
disappear at different timestamps.

\begin{defn}[Temporal network]
  A \emph{temporal network} (or graph) is a tuple
  $G = (V, E, \mathcal{T}, \rho)$, where:
  \begin{itemize}
  \item $V$ is a finite set of nodes,
  \item $E\subseteq V\times V$ is a set of edges,
  \item $\mathbb{T}$ is the \emph{temporal domain} (often taken as
    $\mathbb{N}$ or $\mathbb{R}_+$), and
    $\mathcal{T}\subseteq\mathbb{T}$ is the \emph{lifetime} of the
    network,
  \item $\rho: E\times\mathcal{T}\mapsto\{0,1\}$ is the \emph{presence
      function}, which determines whether an edge is present in the
    network at each timestamp.
  \end{itemize}
  The \emph{available dates} of an edge are the set
  $\mathcal{I}(e) = \{t\in\mathcal{T}: \rho(e,t)=1\}$.
\end{defn}

Temporal networks can also have weighted edges. In this case, it is
possible to have constant weights (edges can only appear or disappear
over time, and always have the same weight), or time-varying
weights. In the latter case, we can set the domain of the presence
function to be $\mathbb{R}_+$ instead of $\{0,1\}$, where by
convention a zero weight corresponds to an absent edge.

\begin{defn}[Additive temporal network]
  A temporal network is said to be \emph{additive} if for all $e\in E$
  and $t\in\mathcal{T}$, if $\rho(e,t)=1$, then
  $\forall t'>t, \rho(e, t') = 1$. Edges can only be added to the
  network, never removed.
\end{defn}

\section{Network partitioning}%
\label{sec:network-partitioning}

\section{Persistent homology for networks}%
\label{sec:pers-homol-netw}

We now consider the problem of applying persistent homology to network
data. An undirected network is already a simplicial complex of
dimension 1. However, this will not be sufficient to capture enough
topological information: we need to introduce higher-dimensional
simplices. The first possible method is to project the network on a
metric space, thus transforming the network data into a point cloud
data. For this, we need to compute the distance between each pair of
nodes in the network (via shortest path distance for instance). This
also requires the network to be connected.

Another usual method for weighted networks is called the \emph{weight
  rank clique filtration} (WRCF), which filters the network based on
weights. The procedure works as follows:
\begin{enumerate}
\item Set the set of all nodes, without any edge, as filtration
  step~0.
\item Rank all edge weights in decreasing order $\{w_1,\ldots,w_n\}$.
\item At filtration step $t$, keep only the edges whose weights are
  less than $w_t$, thus creating an unweighted graph.
\item Define the maximal cliques of the resulting graph to be
  simplices.
\end{enumerate}

At each step of the filtration, we construct a simplicial complex
based on cliques: this is called a \emph{clique complex}. It is
necessarily valid since a subset of a clique is necessarily a clique
itself, and the same is true for the intersection of two cliques.

This leads to a first possibility for applying persistent homology to
temporal networks. It is possible to segment the lifetime of the
network into sliding windows, creating a static graph on each window
by retaining only the edges available during the time interval. We can
then apply WRCF on each static graph in the sequence, obtaining a
filtered complex for each window, to which we can then apply
persistent homology.

This method can quickly become very computationally expensive, as
finding all maximal cliques (using the Bron-Kerbosch algorithm for
example) is a complicated problem in itself. In practice, we often
restrict the search to cliques of dimension lower than a certain bound
$d_M$. With this restriction, the new simplicial complex is
homologically equivalent to the original: they have the same homology
groups up to $H_{d_M-1}$.

This method is sensitive to the choice of sliding windows on the time
scale. The width and the overlap of the windows can completely change
the networks created and their topological features. Too small a
window, and the network becomes too small to have any significant
topological properties, too large, and we lose important information
in the evolution of the network over time.

\section{Zigzag persistence}%
\label{sec:zigzag-persistence}


\backmatter%

\nocite{*}
\bibliographystyle{plain}
\bibliography{}%
\label{cha:bibliography}

\end{document}



%%% Local Variables:
%%% mode: latex
%%% TeX-master: t
%%% End:
