\usepackage{fontspec}

% \setmainfont{Linux Libertine O}
% \setsansfont{Linux Biolinum O}
% \setmonofont[Scale=0.83]{Inconsolata}

\setmainfont{Libertinus Serif}
\setsansfont{Libertinus Sans}
\setmonofont[Scale=0.95]{Inconsolata}

\usepackage{amsmath}
\usepackage{amssymb}
\usepackage{amsthm}
\usepackage{unicode-math}
%\setmathfont{Libertinus Math}
\usepackage[ruled]{algorithm2e}

\newcommand*\diff{\mathop{}\!\mathrm{d}}

\usepackage{polyglossia}
\setdefaultlanguage[variant=british]{english}

\usepackage{graphicx}
\usepackage[dvipsnames]{xcolor}
\usepackage{wrapfig}
\usepackage{caption}
\usepackage{subcaption}
\usepackage{lettrine}

\usepackage{thmtools}

\theoremstyle{plain}
\newtheorem{thm}{Theorem}[chapter]
%\newcommand{\thmautorefname}{theorem}
\newtheorem{lem}[thm]{Lemma}
%\newcommand{\lemautorefname}{lemma}
\newtheorem{cor}[thm]{Corollary}
%\newcommand{\corautorefname}{corollary}
\newtheorem{prop}[thm]{Proposition}
%\newcommand{\propautorefname}{proposition}
\theoremstyle{definition}
\newtheorem{defn}{Definition}[chapter]
%\newcommand{\defnautorefname}{definition}
\newtheorem{expl}{Example}[chapter]
%\newcommand{\explautorefname}{example}
\theoremstyle{remark}
\newtheorem*{rem}{Remark}
\newtheorem*{note}{Note}
\newtheorem*{notation}{Notation}

\DeclareMathOperator{\dgm}{dgm}

\usepackage{tikz-network}
\usepackage{tikz}
\usetikzlibrary{patterns,backgrounds,positioning,chains,lindenmayersystems,intersections}

\usepackage{csquotes}
\usepackage[backend=biber,style=numeric-comp,backref,url=false,minnames=3]{biblatex}
\bibliography{TDA,temporalgraphs,Other}

\usepackage{pdfpages}

\usepackage{microtype}

%% Propriétés du document PDF
\usepackage[unicode,colorlinks=true]{hyperref}

\hypersetup{
  pdfauthor={Dimitri Lozeve},
  pdftitle={Topological Data Analysis of Temporal Networks},
  pdfsubject={A dissertation submitted un partial fulfilment of the degree of Master of Science in Applied Statistics},
  pdfkeywords={dissertation,stats,msc,tda,networks},
  pdfpagemode=UseOutlines,
  pdfpagelayout=TwoColumnRight,
  linkcolor=MidnightBlue,
  filecolor=MidnightBlue,
  urlcolor=MidnightBlue,
  citecolor=MidnightBlue
}

\usepackage{minted}
\setminted{autogobble,fontsize=\footnotesize,linenos,stepnumber=5}

%% Pour la classe memoir /!\

%% Marges
\setlrmarginsandblock{3cm}{3cm}{*}
\setulmarginsandblock{3cm}{3cm}{*}
\checkandfixthelayout%

%% Numérotation des divisions logiques
\setsecnumdepth{subsubsection}
\maxsecnumdepth{subsubsection}

%% Profondeur de la ToC
\settocdepth{subsubsection}
\maxtocdepth{subsubsection}

%% Style des titres des divisions logiques
\setsecheadstyle{\Large\scshape}
\setsubsecheadstyle{\large\scshape}

%% Abstract
\abstractintoc%
\renewcommand{\abstractnamefont}{\normalfont\large\scshape}
\renewcommand{\abstracttextfont}{\normalfont\normalsize}

%% épigraphes
\setlength{\epigraphwidth}{0.5\textwidth}
\epigraphtextposition{flushleftright}

%% Couleurs
%\definecolor{purpletouch}{RGB}{103,30,117}
\definecolor{bleux}{RGB}{0,62,92}
\definecolor{OxfordBlue}{RGB}{0,33,71}

\author{Dimitri Lozeve}
\date{September 2018}
\title{Topological Data Analysis of Temporal Networks}







%%% Local Variables:
%%% mode: latex
%%% TeX-master: "dissertation"
%%% End:
